\documentclass{article}
\usepackage[utf8]{inputenc}
\usepackage{amsfonts}

\title{Kinodynamic Motion Planning}
\author{}

\begin{document}

\maketitle

\section*{Problem statement}

$$
RRT: (X, \theta, f, \epsilon) -> \textnormal{path}
$$

\subsection*{}

$X$ - map. It is splitted into free space and obstacles.

The map is rectangular and is defined by its height and width:
$$
X \in \mathbb{R}^2
$$

Each obstacle is a polygon defined by its vertices:
$$
X_{o_i} \in \{\mathbb{R}^2\}
$$

Thus, free space can be defined as follows:
$$
X_f = X \setminus X_o
$$

\subsection*{}

$\theta$ - turning radius. It is defined as forward velocity divided by maximum angular velocity:

$$
\theta = \frac{v_f}{v_a}
$$

\subsection*{}

$f$ - steering function. A function (ex. Dubins, Reeds-Shepp) taking two points as input and returning either a path or none (if no path can be constructed)

$$
f: \mathbb{R}^2 -> \textnormal{path}
$$

\subsection*{}

$\epsilon$ - precision, error level that will be ignored. It is defined as a tuple with error values for x, y and $\theta$ respectively:

$$
\epsilon = (\epsilon_x, \epsilon_y, \epsilon_\theta)
$$

\subsection*{}

Goal control function $g(p, \epsilon)$ checks if position $p$ can be treated as final within $\epsilon$ precision:
$$g(p, \epsilon)=1_{3}\Big[(\epsilon_{x} - p_{x} \geq 0) + (\epsilon_{y} - p_{y} \geq 0) + (\epsilon_{\theta} - p_{\theta} \geq 0)\Big]$$
\end{document}
